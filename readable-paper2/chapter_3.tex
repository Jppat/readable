%   Filename    : chapter_4.tex 
\chapter{Research Methodology}
This chapter lists and discusses the specific steps and activities that will be performed to accomplish the project.

\section{Research Activities}
\subsection{Acoustic Modelling}
For this paper, we will be utilizing the Kaldi ASR toolkit for modelling the acoustic model. Kaldi is popular when it comes to automatic speech recognition because it is flexible and accessible compared to the other ASR toolkits. 

For the audio file, three speakers will utter two-to-three syllables of Hiligaynon words. These audio files will be grouped and classified as testing and training data. Each audio file has a corresponding transcription text document that will help in achieving an accurate model result. Speakers are also required to record the given words clearly and audibly. Different speakers for the testing and training audio data will be observed to yield an unbiased result.

One important key component for the Kaldi to function is its various phonemes. These phonemes were paired with words from the system’s local dictionary. Table ~\ref{tab:phoneme-hiligaynon} from page ~\pageref{tab:phoneme-hiligaynon} shows the phonemes that will be used for this study's acoustic modelling.

\subsection{Forced Alignment}
\citeauthor{pascual-2017} \citeyear{pascual-2017} implemented an HMM-based Viterbi-forced alignment method to produce a likelihood score that tells whether there are reading miscues to the input speech in reference to the target speech. A threshold-based classification using a threshold likelihood score was used to determine this decision. Similarly, \citeauthor{rasmussen-2009} \citeyear{rasmussen-2009} also implemented a forced-alignment method to detect reading miscues. This shows how time-aligned transcriptions are useful for application related to speech recognition \cite{dimzon-2020}.

In this study, reading miscue detection is aimed to be achieved also by forced alignment method. The Kaldi ASR toolkit includes alignment scripts, which the study aims to use for the mentioned objective.Kaldi's alignment process outputs a sequence of alignment ids which tell what was spoken in a given frame.

This information will then be used as inputs for a logistic regression model to calculate the probability of an utterance being acceptable, in reference to a target speech or audio data. This is similar to the approach of \citeauthor{pascual-2017} \citeyear{pascual-2017}, where their study used threshold-based classification to determine the likelihood of detecting reading miscues.

\subsection{Evaluation}

Evaluation of the system will be done by measuring the word error rate across different iterations of a 5-fold cross validation technique to maximize the items gathered for the dataset.

\section{Calendar of Activities}
%
%  the following commands will be used for filling up the bullets in the Gantt chart
%
\newcommand{\weekone}{\textbullet}
\newcommand{\weektwo}{\textbullet \textbullet}
\newcommand{\weekthree}{\textbullet \textbullet \textbullet}
\newcommand{\weekfour}{\textbullet \textbullet \textbullet \textbullet}

%
%  alternative to bullet is a star 
%
\begin{comment}
   \newcommand{\weekone}{$\star$}
   \newcommand{\weektwo}{$\star \star$}
   \newcommand{\weekthree}{$\star \star \star$}
   \newcommand{\weekfour}{$\star \star \star \star$ }
\end{comment}


Table \ref{tab:timetableactivities} shows a Gantt chart of the activities.  Each bullet represents approximately
one week worth of activity.

\begin{table}[!]   %t means place on top, replace with b if you want to place at the bottom
	\setlength{\extrarowheight}{2pt}
	\setlength{\tabcolsep}{0.2em}
	\centering
	\caption{Timetable of Activities} \vspace{0.25em}
	\begin{tabular}{|p{2in}|c|c|c|c|c|c|c|c|c|c|c|} \hline
		\centering Activities (2009) & Sept & Oct & Nov & Dec & Jan & Feb & Mar & Apr & May & Jun & Jul\\ \hline
		Prerequisite knowledge research & ~~~\weektwo & ~~~\weektwo & & & & & & & & &\\ \hline
		Identification of potential proposal and features &  & \weektwo & \weekthree & & & & & & & &\\ \hline
		Writing of proposal paper     &   &  & \weekone & \weekfour & & & & & & &\\ \hline
		Recording of audio files    & & &  & & & \weektwo~~~ &  & & & &\\ \hline
		Modeling     &   &  &  & & & \weekone & \weekthree & \weekone & & &\\ \hline
		Development of the system &   &  &  &  &  &  & \weekone~~~~~ & \weekthree & \weektwo & \weekone &\\ \hline
		Analyzing and interpretation of the results & & & & & & & & & \weekthree & \weekone &\\ \hline
		Documentation & ~~~\weektwo  & \weekfour & \weekfour & \weekfour &  & \weekthree & \weekfour & \weekthree & \weekfour & \weekone & \weekone \\ \hline
	\end{tabular}
	\label{tab:timetableactivities}
\end{table}

