%   Filename    : chapter_4.tex 
\chapter{Research Methodology}
This chapter lists and discusses the specific steps and activities that will be performed to accomplish the project.

\section{Research Activities}
For this paper, we will be utilizing the Kaldi ASR toolkit. Kaldi is popular when it comes to automatic speech recognition because it is flexible and accessible compared to the other ASR toolkits. For the audio file, three speakers will utter two-to-three syllables of Hiligaynon words. These audio files will be grouped and classified as testing and training data. Each audio file has a corresponding transcription text document that will help in achieving an accurate model result. Speakers are also required to record the given words clearly and audibly. Different speakers for the testing and training audio data will be observed to yield an unbiased result.

One important key component for the Kaldi to function is its various phonemes. These phonemes were paired with words from the system’s local dictionary. The table below shows the phonemes that will be used in this study.

\begin{table}[h]   %t means place on top, replace with b if you want to place at the bottom
	\setlength{\extrarowheight}{2pt}
	\setlength{\tabcolsep}{0.2em}
	\centering
	\caption{Table of Hiligaynon-specific phonemes used in training the system's acoustic model (Gavieta, et al., 2022, p. 20)} \vspace{0.25em}
	\begin{tabular}{|p{2in}|c|c|} \hline
		\centering Phone Class & Phones/Diphone \\ \hline
		Bilabial stops & /p/, /b/ \\ \hline
		Dental stops & /t/, /d/ \\ \hline
		Velar stops & /k/, /g/  \\ \hline
		Africate & j \\ \hline
		Fricatives & /s/, /sh/, /v/, /z/, /f/  \\ \hline
		Nasals & /m/, /n/m /ng/  \\ \hline
		Liquids & /l/, /r/ \\ \hline
		Semivowels/Glides & /w/, /y/  \\ \hline
		Vowels & /i/, /e/, /a/, /o/, /u/ \\ \hline
		Diphones & /ha/, /he/, /hi/, /ho/, /hu/, /at/, /aw/, /ay/, /oy/ \\ \hline
		
		
	\end{tabular}
	\label{tab:phoneme-hiligaynon}
\end{table}


\section{Calendar of Activities}

A Gantt chart showing the schedule of the activities should be included as a table. For example:

Table \ref{tab:timetableactivities} shows a Gantt chart of the activities.  Each bullet represents approximately
one week worth of activity.

%
%  the following commands will be used for filling up the bullets in the Gantt chart
%
\newcommand{\weekone}{\textbullet}
\newcommand{\weektwo}{\textbullet \textbullet}
\newcommand{\weekthree}{\textbullet \textbullet \textbullet}
\newcommand{\weekfour}{\textbullet \textbullet \textbullet \textbullet}

%
%  alternative to bullet is a star 
%
\begin{comment}
   \newcommand{\weekone}{$\star$}
   \newcommand{\weektwo}{$\star \star$}
   \newcommand{\weekthree}{$\star \star \star$}
   \newcommand{\weekfour}{$\star \star \star \star$ }
\end{comment}

\begin{table}[!]   %t means place on top, replace with b if you want to place at the bottom
\setlength{\extrarowheight}{2pt}
\setlength{\tabcolsep}{0.2em}
\centering
\caption{Timetable of Activities} \vspace{0.25em}
\begin{tabular}{|p{2in}|c|c|c|c|c|c|c|c|c|c|c|} \hline
\centering Activities (2009) & Sept & Oct & Nov & Dec & Jan & Feb & Mar & Apr & May & Jun & Jul\\ \hline
Prerequisite knowledge research & ~~~\weektwo & ~~~\weektwo & & & & & & & & &\\ \hline
Identification of potential proposal and features &  & \weektwo & \weekthree & & & & & & & &\\ \hline
Writing of proposal paper     &   &  & \weekone & \weekfour & & & & & & &\\ \hline
Recording of audio files    & & &  & & & \weektwo~~~ &  & & & &\\ \hline
Modeling     &   &  &  & & & \weekone & \weekthree & \weekone & & &\\ \hline
Development of the system &   &  &  &  &  &  & \weekone~~~~~ & \weekthree & \weektwo & \weekone &\\ \hline
Analyzing and interpretation of the results & & & & & & & & & \weekthree & \weekone &\\ \hline
Documentation & ~~~\weektwo  & \weekfour & \weekfour & \weekfour &  & \weekthree & \weekfour & \weekthree & \weekfour & \weekone & \weekone \\ \hline
\end{tabular}
\label{tab:timetableactivities}
\end{table}

