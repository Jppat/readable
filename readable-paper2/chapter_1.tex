%   Filename    : chapter_1.tex 
\chapter{Introduction}
\label{sec:researchdesc}    %labels help you reference sections of your document

\section{Overview of the Current State of Technology}
\label{sec:overview}

The ability to read is a fundamental skill that is necessary for success in many areas of life. Unfortunately, there are many adults in the Philippines who have never learned to read or who have difficulty reading due to various reasons such as illiteracy, limited education, or learning disabilities. These individuals often face significant barriers to employment, education, and social participation, leading to a cycle of poverty and marginalization.

In 2019, the Philippines achieved a literacy rate of 96.5 \% for the segment of the population aged 10 and over according to the PSA’s Functional Literacy, Education and Mass Media Survey (FLEMMS), as reported in an article from Business World entitled, “Literacy rate estimated at 93.8\% among 5 year olds or older — PSA.” Literacy was defined as the ability to read and write “with understanding of simple messages in any language or dialect.” However, the same article notes that this was the same rate observed in 2013, a matter described as alarming by University of Asia and the Pacific Senior Economist Cid L. Terosa, stating that even minimal improvements should be expected especially after six years.

More recently, according to a report published by UNICEF in collaboration with UNESCO and the World Bank, the percentage of 10-year-olds in low- and middle-income countries who are unable to read is as high as 70\%. This figure has likely been affected by school closures brought about by the COVID-19 pandemic. The same report also stated that only 10\% of children in the Philippines were able to read simple text as of March 2022. Alarmingly, a separate report published by the World Bank in 2021 found that the rate of learning poverty - defined as the inability to read simple text by age 10 - in the Philippines was at 90\%. These statistics highlight the urgent need to address the education crisis in the Philippines and the need to further augment the country’s current literacy situation.

To address this problem, we propose the development of an automatic reading miscue detection system called Readable, specifically targeting the Hiligaynon language. Hiligaynon, also known as Ilonggo, is an Austronesian language spoken in the Western Visayas region of the Philippines, particularly in the provinces of Iloilo, Guimaras, Negros Occidental, and Capiz. It is one of the major languages of the Philippines, spoken by millions of people as a first or second language.
Our reading miscue detection system will utilize machine learning techniques including automatic speech recognition (ASR). ASR is a technology that allows computers to automatically recognize and transcribe spoken language, and it has made significant advances in recent years. However, it can still be challenging to achieve high levels of accuracy for some languages and accents, especially those that are underrepresented in ASR training data. By targeting local languages like Hiligaynon and designing our ASR system to work well for these languages, we can help ensure that our reading tutor is accessible and effective for non-reading adults in the Philippines.

While there are some similar applications like Google Read Along available for reading instruction, they may not be accessible or relevant for many non-reading adults in the Philippines due to language barriers or lack of internet connectivity. By targeting local languages like Hiligaynon and utilizing the benefits of natural language processing and machine learning techniques, our automatic reading tutor can provide personalized and effective reading instruction that is accessible and relevant for non-reading adults in the Philippines. By providing accessible, effective, and scalable reading education in Hiligaynon, we hope to improve the lives and prospects of non-reading adults in the Philippines and break the cycle of poverty and illiteracy. Children with strong literacy skills grow more consistently and confidently in their studies, and reading literacy is a crucial gateway to other learning areas such as the humanities, mathematics, and the sciences. By addressing learning poverty and promoting reading literacy, we can help ensure that children in the Philippines have the opportunity to reach their full potential and succeed in their studies.

\section{Problem Statement}

Given the current educational crisis our country is facing \cite{unicef-2022} and with the aim to further improve the current state of  literacy rate of our country \cite{hernandez-2020}, the development of automatic reading tutor systems which entails building reading miscue detection systems and other related programs becomes relevant. Furthermore, the limited resources available for Hiligaynon in the context of speech processing technologies opens a good opportunity to attempt to make a contribution for the said domain of interest.

\section{Research Objectives}
\label{sec:researchobjectives}

\subsection{General Objective}
\label{sec:generalobjective}

This subsection states the over--all goal that must be achieved to answer the problem.
Address the following: Given your research challenge or opportunity, how do you intend  to solve it? What is the output of your research?


\subsection{Specific Objectives}
\label{sec:specificobjectives}

%
%  \begin{comment} ... \end{comment} is used for multiple lines of comment
%

This subsection is an elaboration of the general objective.  
It states the specific steps that must be undertaken to accomplish the general objective.  
These objectives must be \textbf{S}pecific, \textbf{M}easurable, \textbf{A}ttainable, \textbf{R}ealistic, \textbf{T}ime-bounded.  
A specific objective start with ``to $<$verb$>$'' for example: to design/survey/review/analyze.

Studying a particular programming language or development tool (e.g., to study Windows/Object-Oriented/Graphics/C++ programming) to  accomplish the general objective is inherent in all thesis and, therefore, must not be included here.


\begin{comment}
% IPR acknowledgement: the following sentences and examples are from Ethel Ong's slides 
%     on Research Objectives
How to formulate your research objectives:
1. Identify what research steps do you need to perform to achieve your general objective.
2. Identify the questions that must be answered for you to achieve your general objective.
    Thereafter, convert these questions into action statements

Example #1:

Research Question:
  What are the general features of a web-based learning environment?

Specific Objective:
   To review existing web-based learning environment that teaches language learning for children


Example #2:

Research Question:
   How will you represent commonsense knowledge for use by computer systems?

Specific Objective:
   To identify knowledge representation approaches used by existing story generation systems

Example #3:
Research Question:
   What types of storytelling knowledge are needed to generate stories?

Specific Objective:
    To identify the different types of storytelling knowledge used in generating stories

Example #4:
Research Question:
    What machine learning approaches will you utilize?

Specific Objective:
    To determine existing machine learning algorithms [that can be used in training the computer system to detect cyberbullying cases] 

Example #5: Research Question:
    How will your research output be evaluated?

Specific Objective:
    To define evaluation metrics for validating the accuracy of the translation

\end{comment}

%
%  The following are example specific objectives; replace them with your own 
%

\begin{enumerate}
   \item To review related literature, compare and contrast existing algorithms (on what problem?);
   \item To develop a new algorithm (for what purpose?)
   \item To analyze the algorithm (based on what criteria?)
\end{enumerate}


\section{Scope and Limitations of the Research}
\label{sec:scopelimitations}

This section discusses the boundaries (with respect to the objectives) of the research and the constraints within 
which the research will be developed.

\begin{comment}

%
% IPR acknowledgement: the sentences inside this comment are from Ethel Ong's slides on Scope and Limitations of the Research
%
Generally, one paragraph should be allotted for each of your research objectives.

Each paragraph contains a brief overview of the concept/theory and the purpose of doing the associated objective.

Each paragraph also includes a description of the scope/limitation of your study.

* Please refer to the slides for examples.

\end{comment}


\section{Significance of the Research}
\label{sec:significance}

This section explains why research must be done in this area.
 It rationalizes the objective of the research with that of the stated problem. 
 Avoid including sentences such as ``This research will be beneficial to the proponent/department/college'' as this is already an inherent requirement of all BSCS majors.  Focus on the research's contribution to the Computer Science field.

The following are guide questions that may help your formulate the significance of your research. 


%
% IPR acknowledgement: the following list of items are from Ethel Ong's slides on Significance of the Research
%
\begin{itemize}
\item  What is the relevance of your work to the computer science community? 

\begin{itemize} 
\item What will be your technical contributions, in terms of algorithms, or approaches, or new domain? 
\item What is your value-added compared to existing systems? 
\end{itemize}

\item What will be your contributions to society in general? 
    \begin{itemize}
      \item Who will benefit from your system? 
      \item Who are your target users and how will this system benefit them? 
   \end{itemize}
\end{itemize}

\begin{comment}
If applicable, describe possible commercialization and/or innovation in your research.
\end{comment}


